\documentclass{report}
% Change "article" to "report" to get rid of page number on title page
\usepackage{amsmath,amsfonts,amsthm,amssymb}
\usepackage{setspace}
\usepackage{Tabbing}
\usepackage{fancyhdr}
\usepackage{lastpage}
\usepackage{extramarks}
\usepackage{chngpage}
\usepackage{soul,color}
\usepackage{graphicx,float,wrapfig}
\usepackage{lscape}

% In case you need to adjust margins:
\topmargin=-0.45in      %
\evensidemargin=-.50in     %
\oddsidemargin=-.50in      %
\textwidth=7.5in        %
\textheight=9.0in       %
\headsep=0.25in         %

% Homework Specific Information
\newcommand{\hmwkDueDate}{Feb 15, 2012}
\newcommand{\hmwkClass}{ME572}
\newcommand{\hmwkNumber}{4}
\newcommand{\hmwkAuthorName}{Jedediah Frey}

% Setup the header and footer
\pagestyle{fancy}                                                       %
\lhead{\hmwkAuthorName}                                                 %
\chead{\hmwkClass\ Homework \#\hmwkNumber}  %
\rhead{\firstxmark}                                                     %
\lfoot{\lastxmark}                                                      %
\cfoot{}                                                                %
\rfoot{Page\ \thepage\ of\ \pageref{LastPage}}                          %
\renewcommand\headrulewidth{0.4pt}                                      %
\renewcommand\footrulewidth{0.4pt}                                      %

% This is used to trace down (pin point) problems
% in latexing a document:
%\tracingall

%%%%%%%%%%%%%%%%%%%%%%%%%%%%%%%%%%%%%%%%%%%%%%%%%%%%%%%%%%%%%
% Some tools
\newcommand{\parens}[1]{\left(#1\right)}
\newcommand{\abs}[1]{\lvert#1\rvert}
\newcommand{\norm}[1]{\left|\left|#1\right|\right|}
\newcommand{\bs}[1]{\boldsymbol{#1}}

\newcommand{\enterProblemHeader}[1]{\nobreak\extramarks{#1}{#1 continued on next page\ldots}\nobreak%
                                    \nobreak\extramarks{#1 (continued)}{#1 continued on next page\ldots}\nobreak}%
\newcommand{\exitProblemHeader}[1]{\nobreak\extramarks{#1 (continued)}{#1 continued on next page\ldots}\nobreak%
                                   \nobreak\extramarks{#1}{}\nobreak}%

\newlength{\labelLength}
\newcommand{\labelAnswer}[2]
  {\settowidth{\labelLength}{#1}%
   \addtolength{\labelLength}{0.25in}%
   \changetext{}{-\labelLength}{}{}{}%
   \noindent\fbox{\begin{minipage}[c]{\columnwidth}#2\end{minipage}}%
   \marginpar{\fbox{#1}}%

   % We put the blank space above in order to make sure this
   % \marginpar gets correctly placed.
   \changetext{}{+\labelLength}{}{}{}}%

\setcounter{secnumdepth}{0}
\newcommand{\homeworkProblemName}{}%
\newcounter{homeworkProblemCounter}%
\newenvironment{homeworkProblem}[1][Problem \arabic{homeworkProblemCounter}]%
  {\stepcounter{homeworkProblemCounter}%
   \renewcommand{\homeworkProblemName}{#1}%
   \section{\homeworkProblemName}%
   \enterProblemHeader{\homeworkProblemName}}%
  {\exitProblemHeader{\homeworkProblemName}}%

\newcommand{\problemAnswer}[1]
  {\noindent\fbox{\begin{minipage}[c]{\columnwidth}#1\end{minipage}}}%

\newcommand{\problemLAnswer}[1]
  {\labelAnswer{\homeworkProblemName}{#1}}

\newcommand{\homeworkSectionName}{}%
\newlength{\homeworkSectionLabelLength}{}%
\newenvironment{homeworkSection}[1]%
  {% We put this space here to make sure we're not connected to the above.
   % Otherwise the changetext can do funny things to the other margin

   \renewcommand{\homeworkSectionName}{#1}%
   \settowidth{\homeworkSectionLabelLength}{\homeworkSectionName}%
   \addtolength{\homeworkSectionLabelLength}{0.25in}%
   \changetext{}{-\homeworkSectionLabelLength}{}{}{}%
   \subsection{\homeworkSectionName}%
   \enterProblemHeader{\homeworkProblemName\ [\homeworkSectionName]}}%
  {\enterProblemHeader{\homeworkProblemName}%

   % We put the blank space above in order to make sure this margin
   % change doesn't happen too soon (otherwise \sectionAnswer's can
   % get ugly about their \marginpar placement.
   \changetext{}{+\homeworkSectionLabelLength}{}{}{}}%

\newcommand{\sectionAnswer}[1]
  {% We put this space here to make sure we're disconnected from the previous
   % passage

   \noindent\fbox{\begin{minipage}[c]{\columnwidth}#1\end{minipage}}%
   \enterProblemHeader{\homeworkProblemName}\exitProblemHeader{\homeworkProblemName}%
   \marginpar{\fbox{\homeworkSectionName}}%

   % We put the blank space above in order to make sure this
   % \marginpar gets correctly placed.
   }%

%%%%%%%%%%%%%%%%%%%%%%%%%%%%%%%%%%%%%%%%%%%%%%%%%%%%%%%%%%%%%


%%%%%%%%%%%%%%%%%%%%%%%%%%%%%%%%%%%%%%%%%%%%%%%%%%%%%%%%%%%%%
% Make title
\title{\vspace{2in}\textmd{\textbf{\hmwkClass:\ Homework \#\hmwkNumber}}\\\normalsize\vspace{0.1in}\small{Due\ on\ \hmwkDueDate}}
\date{}
\author{\textbf{\hmwkAuthorName}}
%%%%%%%%%%%%%%%%%%%%%%%%%%%%%%%%%%%%%%%%%%%%%%%%%%%%%%%%%%%%%

\begin{document}
\begin{spacing}{1.1}
\maketitle
\newpage
\clearpage
\begin{homeworkProblem}
\textbf{Device 1:}
\begin{eqnarray}
P_x&=&L_3\parens{\cos(\theta_2)\sin(\theta_1)\sin(\theta_3)+\cos(\theta_3)\sin(\theta_1)\sin(\theta_2)}+d1\cos(\theta_1)+L_2\sin(\theta_1)\sin(\theta_2) \\
&=& d_1\cos(\theta_1) + L_3\sin(\theta_1)\sin(\theta_2 + \theta_3) + L_2\sin(\theta_1)\sin(\theta_2) \\
&=& d_1\cos(\theta_1) + \sin(\theta_1)(L_3\sin(\theta_2 + \theta_3) + L_2\sin(\theta_2)) \\
P_y&=&d_1\sin(\theta_1)-L_3(\cos(\theta_1)\cos(\theta_2)\sin(\theta_3)+\cos(\theta_1)\cos(\theta_3)\sin(\theta_2))-L_2\cos(\theta_1)\sin(\theta_2) \\
&=&     d_1\sin(\theta_1) - L_3\cos(\theta_1)\sin(\theta_2 + \theta_3)-L_2\cos(\theta_1)\sin(\theta_2) \\
&=&     d_1\sin(\theta_1) - \cos(\theta_1)(L_3\sin(\theta_2 + \theta_3)+L_2\sin(\theta_2)) \\
P_z&=&L_1+L_3\cos(\theta_2+\theta_3)+L_2\cos(\theta_2) \\
T_{w}&=&\begin{bmatrix} 
\cos(\theta_1)&-\cos(\theta_2+\theta_3)\sin(\theta_1)& -\sin(\theta_2+\theta_3)\sin(\theta_1) & P_x \\
\sin(\theta_1)&\cos(\theta_2+\theta_3)\cos(\theta_1) &    \sin(\theta_2+\theta_3)\sin(\theta_1) & P_y \\
0&\sin(\theta_2+\theta_3)&\cos(\theta_2+\theta_3)& P_z \\
0&0&0&1\end{bmatrix}
\end{eqnarray}
\textbf{Device 2:}
\begin{eqnarray}
P_x&=& d_1\cos(\theta_1)+S_3\sin(\theta_1)\sin(\theta_2) \label{Px2}\\
P_y&=& d_1\sin(\theta_1)-S_3\cos(\theta_1)\sin(\theta_2) \label{Py2}\\
P_z&=&L_1 + S_3\cos(\theta_2) \label{Pz2}\\
T_{w}&=&\begin{bmatrix}
\cos(\theta_1) & -\cos(\theta_2)\sin(\theta_1) & \sin(\theta_1)\sin(\theta_2)&P_x \\
\sin(\theta_1) & \cos(\theta_1)\cos(\theta_2) & -\cos(\theta_1)\sin(\theta_2)& P_y \\
0 & \sin(\theta_2) & \cos(\theta_2) & P_z  \\
0 & 0 & 0 & 1\end{bmatrix}
\end{eqnarray}
\end{homeworkProblem}
\newpage
\begin{homeworkProblem}
\textbf{Device 2:}

[Completed first since it was simpler]


Solve for $S_3$ by squaring and adding all of the terms. Subtract $L_1$ from the $P_z$ equation first.
\begin{eqnarray}
P_x^2+P_y^2+(P_z-L_1)^2&=&S_3^2 + d_1^2 \\ 
S_3^2&=&{P_x^2 + P_y^2 + (P_z-L_1)^2- d_1^2} \\
S_3&=&\pm\sqrt{P_x^2 + P_y^2 + (P_z-L_1)^2- d_1^2} 
\end{eqnarray}
Given the world transform looks like:
\begin{eqnarray}
T_{w}&=&\begin{bmatrix}
N_x & O_x & A_x & P_x \\
N_y & O_y & A_y & P_y \\
N_z & O_z & A_z & P_z \\
0 & 0 & 0 & 1
\end{bmatrix}
\end{eqnarray}
The desired variables can be isolated by manipulating the transforms. (Easier to me than trig identities). 
\begin{eqnarray}
\phi_2T_2\phi_3T_3&=&(\phi_1T_1)^{-1}Tw\\
P_x\cos(\theta_1) + P_y\sin(\theta_1)&=&d_1 
\end{eqnarray}
Basic definitions to substitute into the above equation.
\begin{eqnarray}
R&=&\sqrt{P_x^2+P_y^2} \\
\phi&=&tan^{-1}\parens{\frac{P_y}{P_x}} \\
P_x&=&R\cos{\phi} \\
P_y&=&R\sin{\phi}
\end{eqnarray}
Values substituted in and reduced to simplest form.
\begin{eqnarray}
R\cos(\phi)\cos(\theta_1) + R\sin(\phi)\sin(\theta_1)&=&d_1 \\
R\cos(\phi - \theta_1)&=&d_1
\end{eqnarray}
Simplify the angle into $\alpha$ for now.
\begin{eqnarray}
\phi - \theta_1&=&\alpha \\
R\cos(\alpha)&=&d_1
\end{eqnarray}
Multiply both sides by $\sin$ until we end up with $\tan$ on one side.
\begin{eqnarray}
R\cos(\alpha)\sin(\alpha)&=&d_1\sin(\alpha) \\
\frac{R\cos(\alpha)\sin(\alpha)}{d_1}&=&\sin(\alpha) \\
\frac{R\sin(\alpha)}{d_1}&=&\frac{\sin(\alpha)}{\cos(\alpha)} \\
\frac{R\sin(\alpha)}{d_1}&=&\tan{\alpha}
\end{eqnarray}
More trig identities so we can substitute in what we already know.
\begin{eqnarray}
\sin{\alpha}^2+\cos{\alpha}^2&=&1 \\
\sin{\alpha} &=& \sqrt{1-\cos{\alpha}^2} \\
&=&\sqrt{1-\frac{d_1^2}{R^2}}
\end{eqnarray}
Solve for the angle, replace with our definition for $\alpha$. Final symbolic equation is in Eqn. \ref{finalTheta1Dev2}.
\begin{eqnarray}
\alpha&=&\tan^{-1}\parens{\frac{\pm\sqrt{\parens{1-\frac{d_1^2}{P_x^2+P_y^2}}\parens{P_x^2+P_y^2}}}{d_1}} \\
\phi - \theta_1&=& \\
\theta_1&=&\tan^{-1}\parens{\frac{P_y}{P_x}}-\tan^{-1}\parens{\frac{\pm\sqrt{\parens{1-\frac{d_1^2}{P_x^2+P_y^2}}\parens{P_x^2+P_y^2}}}{d_1}} \label{finalTheta1Dev2}
\end{eqnarray}

From the same transform that was done to $T_w$ above, the $P_x$ and $P_y$ equations become:
\begin{eqnarray}
P_y\cos(\theta_1) - Px\sin(\theta_1)&=&-S_3\sin(\theta_2) \\
P_z&=&L_1 + S_3\cos(\theta_2)\\
\sin(\theta_2)&=&\frac{Px\sin(\theta_1)-P_y\cos(\theta_1)}{S_3} \\
\cos(\theta_2)&=&\frac{P_z-L_1}{S_3} \\
\tan(\theta_2)&=&\frac{\frac{Px\sin(\theta_1)-P_y\cos(\theta_1)}{S_3}}{\frac{P_z-L_1}{S_3}}
\end{eqnarray}
One thing to note on the above equation, $S_3$ can not be canceled since it puts the solution in the right quadrant when using atan2 (the only real $tan^{-1}$ there is). \\
\textbf{Device 1:}
$\theta_1$ is solved in the same manner as above. The equation below is the same so the solutions for $\theta_1$ will be too.
\begin{eqnarray}
P_x\cos(\theta_1) + P_y\sin(\theta_1)&=&d_1 \\
\theta_1&=&\tan^{-1}\parens{\frac{P_y}{P_x}}-\tan^{-1}\parens{\frac{\pm\sqrt{\parens{1-\frac{d_1^2}{P_x^2+P_y^2}}\parens{P_x^2+P_y^2}}}{d_1}} 
\end{eqnarray}
\end{homeworkProblem}

\begin{homeworkProblem}
\textbf{Device 1:} \\
\textbf{Device 2:}
There are 2 unique solutions for both $S_3$ and $\theta_1$ and thus 4 solutions to the problem. \\
\begin{tabular}{ l l l }
$S_3=+\sqrt{43}$&$\theta_1=254.2776^o$&$\theta_3=319.6844^o$\\
$S_3=-\sqrt{43}$&$\theta_1=254.2776^o$&$\theta_3=139.6844^o$\\
$S_3=+\sqrt{43}$&$\theta_1=347.6500^o$&$\theta_3=40.3156^o$\\
$S_3=-\sqrt{43}$&$\theta_1=347.6500^o$&$\theta_3=220.3156^o$\\
\end{tabular}
\end{homeworkProblem}
\end{spacing}
\end{document}