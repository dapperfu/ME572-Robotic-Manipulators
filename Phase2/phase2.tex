\documentclass{report}
% Change "article" to "report" to get rid of page number on title page
\usepackage{amsmath,amsfonts,amsthm,amssymb}
\usepackage{setspace}
\usepackage{Tabbing}
\usepackage{fancyhdr}
\usepackage{lastpage}
\usepackage{extramarks}
\usepackage{chngpage}
\usepackage{soul,color}
\usepackage{graphicx}
\usepackage{float,wrapfig}
\usepackage[table]{xcolor}
\usepackage{booktabs}
\usepackage{epstopdf}
\usepackage[toc,page]{appendix}
\usepackage[framed,numbered,autolinebreaks,useliterate]{mcode}

\usepackage{enumerate}
% In case you need to adjust margins:
\topmargin=-0.45in      %
\evensidemargin=0in     %
\oddsidemargin=0in    %
\textwidth=6.5in        %
\textheight=9.0in       %
\headsep=0.25in         %

% Homework Specific Information
\newcommand{\hmwkDueDate}{April 20, 2012}
\newcommand{\hmwkClass}{ME572}
\newcommand{\hmwkNumber}{2}
\newcommand{\hmwkAuthorName}{}

% Setup the header and footer
\pagestyle{fancy}                                                       %
\lhead{\hmwkAuthorName}                                                 %
\chead{\hmwkClass\ Phase \#\hmwkNumber}  %
\rhead{\firstxmark}                                                     %
\lfoot{\lastxmark}                                                      %
\cfoot{}                                                                %
\rfoot{Page\ \thepage\ of\ \pageref{LastPage}}                          %t
\renewcommand\headrulewidth{0.4pt}                                      %
\renewcommand\footrulewidth{0.4pt}                                      %

% This is used to trace down (pin point) problems
% in latexing a document:
%\tracingall

%%%%%%%%%%%%%%%%%%%%%%%%%%%%%%%%%%%%%%%%%%%%%%%%%%%%%%%%%%%%%
% Some tools
\newcommand{\parens}[1]{\left(#1\right)}
\newcommand{\abs}[1]{\lvert#1\rvert}
\newcommand{\norm}[1]{\left|\left|#1\right|\right|}
\newcommand{\bs}[1]{\boldsymbol{#1}}

\newcommand{\enterProblemHeader}[1]{\nobreak\extramarks{#1}{#1 continued on next page\ldots}\nobreak%
                                    \nobreak\extramarks{#1 (continued)}{#1 continued on next page\ldots}\nobreak}%
\newcommand{\exitProblemHeader}[1]{\nobreak\extramarks{#1 (continued)}{#1 continued on next page\ldots}\nobreak%
                                   \nobreak\extramarks{#1}{}\nobreak}%

\newlength{\labelLength}
\newcommand{\labelAnswer}[2]
  {\settowidth{\labelLength}{#1}%
   \addtolength{\labelLength}{0.25in}%
   \changetext{}{-\labelLength}{}{}{}%
   \noindent\fbox{\begin{minipage}[c]{\columnwidth}#2\end{minipage}}%
   \marginpar{\fbox{#1}}%

   % We put the blank space above in order to make sure this
   % \marginpar gets correctly placed.
   \changetext{}{+\labelLength}{}{}{}}%

\setcounter{secnumdepth}{0}
\newcommand{\homeworkProblemName}{}%
\newcounter{homeworkProblemCounter}%
\newenvironment{homeworkProblem}[1][Problem \arabic{homeworkProblemCounter}]%
  {\stepcounter{homeworkProblemCounter}%
   \renewcommand{\homeworkProblemName}{#1}%
   \section{\homeworkProblemName}%
   \enterProblemHeader{\homeworkProblemName}}%
  {\exitProblemHeader{\homeworkProblemName}}%

\newcommand{\problemAnswer}[1]
  {\noindent\fbox{\begin{minipage}[c]{\columnwidth}#1\end{minipage}}}%

\newcommand{\problemLAnswer}[1]
  {\labelAnswer{\homeworkProblemName}{#1}}

\newcommand{\homeworkSectionName}{}%
\newlength{\homeworkSectionLabelLength}{}%
\newenvironment{homeworkSection}[1]%
  {% We put this space here to make sure we're not connected to the above.
   % Otherwise the changetext can do funny things to the other margin

   \renewcommand{\homeworkSectionName}{#1}%
   \settowidth{\homeworkSectionLabelLength}{\homeworkSectionName}%
   \addtolength{\homeworkSectionLabelLength}{0.25in}%
   \changetext{}{-\homeworkSectionLabelLength}{}{}{}%
   \subsection{\homeworkSectionName}%
   \enterProblemHeader{\homeworkProblemName\ [\homeworkSectionName]}}%
  {\enterProblemHeader{\homeworkProblemName}%

   % We put the blank space above in order to make sure this margin
   % change doesn't happen too soon (otherwise \sectionAnswer's can
   % get ugly about their \marginpar placement.
   \changetext{}{+\homeworkSectionLabelLength}{}{}{}}%

\newcommand{\fig}[2]{\begin{figure}[H]
  \centering
    \includegraphics[height=.6\textwidth]{#1}
  \caption{#2}
  \label{EndPosition}
\end{figure}}
%%%%%%%%%%%%%%%%%%%%%%%%%%%%%%%%%%%%%%%%%%%%%%%%%%%%%%%%%%%%%


%%%%%%%%%%%%%%%%%%%%%%%%%%%%%%%%%%%%%%%%%%%%%%%%%%%%%%%%%%%%%
% Make title
\title{\vspace{2in}\textmd{\textbf{\hmwkClass:\ Phase \#\hmwkNumber}}\\\normalsize\vspace{0.1in}\small{Due\ on\ \hmwkDueDate}}
\date{}
\author{\textbf{\hmwkAuthorName}}
%%%%%%%%%%%%%%%%%%%%%%%%%%%%%%%%%%%%%%%%%%%%%%%%%%%%%%%%%%%%%

\begin{document}
\begin{spacing}{1.1}
\maketitle
\newpage
% Uncomment the \tableofcontents and \newpage lines to get a Contents page
% Uncomment the \setcounter line as well if you do NOT want subsections
%       listed in Contents
%\setcounter{tocdepth}{1}
%\tableofcontents
%\newpage

% When problems are long, it may be desirable to put a \newpage or a
% \clearpage before each homeworkProblem environment

\clearpage

RR robot from homework \#7. The analysis is the same as for HW 7. The peak joint velocity is in joint 1 at position 12.

Symbolic jacobian matrix for RR.
\begin{eqnarray}
J(1,1)&=&\begin{bmatrix}
- 20\cdot \sin(\theta_1 + \theta_2) - 20\cdot \sin(\theta_1)\\ 
\end{bmatrix}
 \nonumber \\J(1,2)&=&\begin{bmatrix}
-20\cdot \sin(\theta_1 + \theta_2)\\ 
\end{bmatrix}
 \nonumber \\J(2,1)&=&\begin{bmatrix}
20\cdot \cos(\theta_1 + \theta_2) + 20\cdot \cos(\theta_1)\\ 
\end{bmatrix}
 \nonumber \\J(2,2)&=&\begin{bmatrix}
20\cdot \cos(\theta_1 + \theta_2)\\ 
\end{bmatrix}
 \nonumber \\J(3,1)&=&\begin{bmatrix}
0\\ 
\end{bmatrix}
 \nonumber \\J(3,2)&=&\begin{bmatrix}
0\\ 
\end{bmatrix}
 \nonumber \\J(4,1)&=&\begin{bmatrix}
0\\ 
\end{bmatrix}
 \nonumber \\J(4,2)&=&\begin{bmatrix}
0\\ 
\end{bmatrix}
 \nonumber \\J(5,1)&=&\begin{bmatrix}
0\\ 
\end{bmatrix}
 \nonumber \\J(5,2)&=&\begin{bmatrix}
0\\ 
\end{bmatrix}
 \nonumber \\J(6,1)&=&\begin{bmatrix}
1\\ 
\end{bmatrix}
 \nonumber \\J(6,2)&=&\begin{bmatrix}
1\\ 
\end{bmatrix}
 \nonumber\end{eqnarray}

Joint position and velocities for trajectory1.dat.
\begin{verbatim}
21 % Number of positions
%  t(s)    Theta1   Theta2   Omega1   Omega2
   0.000   81.469  197.254    0.000    0.000
   0.025   81.672  196.852    0.284   -0.562
   0.050   82.282  195.646    0.568   -1.122
   0.075   83.097  194.041    0.569   -1.120
   0.100   83.913  192.439    0.571   -1.118
   0.125   84.732  190.839    0.574   -1.116
   0.150   85.557  189.241    0.579   -1.115
   0.175   86.392  187.645    0.588   -1.114
   0.200   87.246  186.051    0.606   -1.113
   0.225   88.140  184.458    0.648   -1.112
   0.250   89.140  182.865    0.778   -1.111
   0.275   90.652  181.274    1.680   -1.111
   0.300  264.697  180.320   17.302    1.107
   0.325  268.186  181.910   -0.056    1.111
   0.350  267.780  183.502   -0.407    1.112
   0.375  267.130  185.095   -0.486    1.112
   0.400  266.410  186.688   -0.516    1.113
   0.425  265.660  188.283   -0.530    1.114
   0.450  264.894  189.880   -0.539    1.115
   0.475  264.312  191.079   -0.272    0.558
   0.500  264.118  191.478    0.000    0.000
\end{verbatim}

\fig{trajectory1_joint1}{RR Joint 1 Angle and Velocity}
\fig{trajectory1_joint2}{RR Joint 2 Angle and Velocity}
\fig{trajectory1_EECS}{Position of the end effector for each of the coordinates vs time}
\fig{trajectory1_EECS_3D}{RR end effector position in 3D}
\newpage
World transforms of the robot at positions calculated for coordinate interpolated motion.
\begin{verbatim}
POSITION 1: INPUT: JOINT VARIABLES (  81.469, 197.254,)
OUTPUT:     0.152     0.988     0.000     6.000
           -0.988     0.152     0.000     0.010
            0.000     0.000     1.000     0.000
            0.000     0.000     0.000     1.000
\end{verbatim} \pagebreak[1]\begin{verbatim}
POSITION 2: INPUT: JOINT VARIABLES (  81.672, 196.852,)
OUTPUT:     0.148     0.989     0.000     5.861
           -0.989     0.148     0.000     0.010
            0.000     0.000     1.000     0.000
            0.000     0.000     0.000     1.000
\end{verbatim} \pagebreak[1]\begin{verbatim}
POSITION 3: INPUT: JOINT VARIABLES (  82.282, 195.646,)
OUTPUT:     0.138     0.990     0.000     5.444
           -0.990     0.138     0.000     0.010
            0.000     0.000     1.000     0.000
            0.000     0.000     0.000     1.000
\end{verbatim} \pagebreak[1]\begin{verbatim}
POSITION 4: INPUT: JOINT VARIABLES (  83.097, 194.041,)
OUTPUT:     0.124     0.992     0.000     4.889
           -0.992     0.124     0.000     0.010
            0.000     0.000     1.000     0.000
            0.000     0.000     0.000     1.000
\end{verbatim} \pagebreak[1]\begin{verbatim}
POSITION 5: INPUT: JOINT VARIABLES (  83.913, 192.439,)
OUTPUT:     0.111     0.994     0.000     4.333
           -0.994     0.111     0.000     0.010
            0.000     0.000     1.000     0.000
            0.000     0.000     0.000     1.000
\end{verbatim} \pagebreak[1]\begin{verbatim}
POSITION 6: INPUT: JOINT VARIABLES (  84.732, 190.839,)
OUTPUT:     0.097     0.995     0.000     3.778
           -0.995     0.097     0.000     0.010
            0.000     0.000     1.000     0.000
            0.000     0.000     0.000     1.000
\end{verbatim} \pagebreak[1]\begin{verbatim}
POSITION 7: INPUT: JOINT VARIABLES (  85.557, 189.241,)
OUTPUT:     0.084     0.996     0.000     3.222
           -0.996     0.084     0.000     0.010
            0.000     0.000     1.000     0.000
            0.000     0.000     0.000     1.000
\end{verbatim} \pagebreak[1]\begin{verbatim}
POSITION 8: INPUT: JOINT VARIABLES (  86.392, 187.645,)
OUTPUT:     0.070     0.998     0.000     2.667
           -0.998     0.070     0.000     0.010
            0.000     0.000     1.000     0.000
            0.000     0.000     0.000     1.000
\end{verbatim} \pagebreak[1]\begin{verbatim}
POSITION 9: INPUT: JOINT VARIABLES (  87.246, 186.051,)
OUTPUT:     0.058     0.998     0.000     2.111
           -0.998     0.058     0.000     0.010
            0.000     0.000     1.000     0.000
            0.000     0.000     0.000     1.000
\end{verbatim} \pagebreak[1]\begin{verbatim}
POSITION 10: INPUT: JOINT VARIABLES (  88.140, 184.458,)
OUTPUT:     0.045     0.999     0.000     1.556
           -0.999     0.045     0.000     0.010
            0.000     0.000     1.000     0.000
            0.000     0.000     0.000     1.000
\end{verbatim} \pagebreak[1]\begin{verbatim}
POSITION 11: INPUT: JOINT VARIABLES (  89.140, 182.865,)
OUTPUT:     0.035     0.999     0.000     1.000
           -0.999     0.035     0.000     0.010
            0.000     0.000     1.000     0.000
            0.000     0.000     0.000     1.000
\end{verbatim} \pagebreak[1]\begin{verbatim}
POSITION 12: INPUT: JOINT VARIABLES (  90.652, 181.274,)
OUTPUT:     0.034     0.999     0.000     0.444
           -0.999     0.034     0.000     0.010
            0.000     0.000     1.000     0.000
            0.000     0.000     0.000     1.000
\end{verbatim} \pagebreak[1]\begin{verbatim}
POSITION 13: INPUT: JOINT VARIABLES ( 264.697, 180.320,)
OUTPUT:     0.087    -0.996     0.000    -0.111
            0.996     0.087     0.000     0.010
            0.000     0.000     1.000     0.000
            0.000     0.000     0.000     1.000
\end{verbatim} \pagebreak[1]\begin{verbatim}
POSITION 14: INPUT: JOINT VARIABLES ( 268.186, 181.910,)
OUTPUT:    -0.002    -1.000     0.000    -0.667
            1.000    -0.002     0.000     0.010
            0.000     0.000     1.000     0.000
            0.000     0.000     0.000     1.000
\end{verbatim} \pagebreak[1]\begin{verbatim}
POSITION 15: INPUT: JOINT VARIABLES ( 267.780, 183.502,)
OUTPUT:    -0.022    -1.000     0.000    -1.222
            1.000    -0.022     0.000     0.010
            0.000     0.000     1.000     0.000
            0.000     0.000     0.000     1.000
\end{verbatim} \pagebreak[1]\begin{verbatim}
POSITION 16: INPUT: JOINT VARIABLES ( 267.130, 185.095,)
OUTPUT:    -0.039    -0.999     0.000    -1.778
            0.999    -0.039     0.000     0.010
            0.000     0.000     1.000     0.000
            0.000     0.000     0.000     1.000
\end{verbatim} \pagebreak[1]\begin{verbatim}
POSITION 17: INPUT: JOINT VARIABLES ( 266.410, 186.688,)
OUTPUT:    -0.054    -0.999     0.000    -2.333
            0.999    -0.054     0.000     0.010
            0.000     0.000     1.000     0.000
            0.000     0.000     0.000     1.000
\end{verbatim} \pagebreak[1]\begin{verbatim}
POSITION 18: INPUT: JOINT VARIABLES ( 265.660, 188.283,)
OUTPUT:    -0.069    -0.998     0.000    -2.889
            0.998    -0.069     0.000     0.010
            0.000     0.000     1.000     0.000
            0.000     0.000     0.000     1.000
\end{verbatim} \pagebreak[1]\begin{verbatim}
POSITION 19: INPUT: JOINT VARIABLES ( 264.894, 189.880,)
OUTPUT:    -0.083    -0.997     0.000    -3.444
            0.997    -0.083     0.000     0.010
            0.000     0.000     1.000     0.000
            0.000     0.000     0.000     1.000
\end{verbatim} \pagebreak[1]\begin{verbatim}
POSITION 20: INPUT: JOINT VARIABLES ( 264.312, 191.079,)
OUTPUT:    -0.094    -0.996     0.000    -3.861
            0.996    -0.094     0.000     0.010
            0.000     0.000     1.000     0.000
            0.000     0.000     0.000     1.000
\end{verbatim} \pagebreak[1]\begin{verbatim}
POSITION 21: INPUT: JOINT VARIABLES ( 264.118, 191.478,)
OUTPUT:    -0.098    -0.995     0.000    -4.000
            0.995    -0.098     0.000     0.010
            0.000     0.000     1.000     0.000
            0.000     0.000     0.000     1.000
\end{verbatim} \pagebreak[1]
Manipulability of the robot.
 \pagebreak[1]\begin{verbatim}
%  t(s)       ax       by
   0.000   5.926   20.022
   0.025   5.792   20.020
   0.050   5.390   20.015
   0.075   4.850   20.009
   0.100   4.307   20.006
   0.125   3.760   20.003
   0.150   3.211   20.002
   0.175   2.661   20.001
   0.200   2.108   20.000
   0.225   1.554   20.000
   0.250   1.000   20.000
   0.275   0.445   20.000
   0.300   0.112   20.000
   0.325   0.667   20.000
   0.350   1.222   20.000
   0.375   1.776   20.000
   0.400   2.329   20.000
   0.425   2.881   20.001
   0.450   3.431   20.002
   0.475   3.842   20.004
   0.500   3.979   20.004
\end{verbatim}

\fig{trajectory1_ell1}{Manipulability of RR at position 0.}
\fig{trajectory1_ell13}{Manipulability of RR at position 12.}
\fig{trajectory1_ell21}{Manipulability of RR at position 20.}

\newpage
RRR robot from homework \#4. The peak joint velocity is in joint 2 at position 10.

Symbolic jacobian matrix for RRR.
\begin{eqnarray}
J(1,1)&=&\begin{bmatrix}
5\cdot \cos(\theta_1)\cdot \sin(\theta_2) - 4\cdot \sin(\theta_1) + 4\cdot \cos(\theta_1)\cdot \cos(\theta_2)\cdot \sin(\theta_3) + 4\cdot \cos(\theta_1)\cdot \cos(\theta_3)\cdot \sin(\theta_2)\\ 
\end{bmatrix}
 \nonumber \\J(1,2)&=&\begin{bmatrix}
\sin(\theta_1)\cdot (4\cdot \cos(\theta_2 + \theta_3) + 5\cdot \cos(\theta_2))\\ 
\end{bmatrix}
 \nonumber \\J(1,3)&=&\begin{bmatrix}
4\cdot \cos(\theta_2 + \theta_3)\cdot \sin(\theta_1)\\ 
\end{bmatrix}
 \nonumber \\J(2,1)&=&\begin{bmatrix}
4\cdot \cos(\theta_1) + 5\cdot \sin(\theta_1)\cdot \sin(\theta_2) + 4\cdot \cos(\theta_2)\cdot \sin(\theta_1)\cdot \sin(\theta_3) + 4\cdot \cos(\theta_3)\cdot \sin(\theta_1)\cdot \sin(\theta_2)\\ 
\end{bmatrix}
 \nonumber \\J(2,2)&=&\begin{bmatrix}
-\cos(\theta_1)\cdot (4\cdot \cos(\theta_2 + \theta_3) + 5\cdot \cos(\theta_2))\\ 
\end{bmatrix}
 \nonumber \\J(2,3)&=&\begin{bmatrix}
-4\cdot \cos(\theta_2 + \theta_3)\cdot \cos(\theta_1)\\ 
\end{bmatrix}
 \nonumber \\J(3,1)&=&\begin{bmatrix}
0\\ 
\end{bmatrix}
 \nonumber \\J(3,2)&=&\begin{bmatrix}
- 4\cdot \sin(\theta_2 + \theta_3) - 5\cdot \sin(\theta_2)\\ 
\end{bmatrix}
 \nonumber \\J(3,3)&=&\begin{bmatrix}
-4\cdot \sin(\theta_2 + \theta_3)\\ 
\end{bmatrix}
 \nonumber \\J(4,1)&=&\begin{bmatrix}
0\\ 
\end{bmatrix}
 \nonumber \\J(4,2)&=&\begin{bmatrix}
\cos(\theta_1)\\ 
\end{bmatrix}
 \nonumber \\J(4,3)&=&\begin{bmatrix}
\cos(\theta_1)\\ 
\end{bmatrix}
 \nonumber \\J(5,1)&=&\begin{bmatrix}
0\\ 
\end{bmatrix}
 \nonumber \\J(5,2)&=&\begin{bmatrix}
\sin(\theta_1)\\ 
\end{bmatrix}
 \nonumber \\J(5,3)&=&\begin{bmatrix}
\sin(\theta_1)\\ 
\end{bmatrix}
 \nonumber \\J(6,1)&=&\begin{bmatrix}
1\\ 
\end{bmatrix}
 \nonumber \\J(6,2)&=&\begin{bmatrix}
0\\ 
\end{bmatrix}
 \nonumber \\J(6,3)&=&\begin{bmatrix}
0\\ 
\end{bmatrix}
 \nonumber\end{eqnarray}

Joint position and velocities for trajectory2.dat.
\begin{verbatim}
17 % Number of positions
%  t(s)    Theta1   Theta2   Theta3   Omega1   Omega2   Omega3
   0.000    0.000  326.976   50.858    0.000    0.000    0.000
   0.015  359.861  324.008   57.465   -0.331   -6.535   14.560
   0.030  359.406  316.558   74.095   -0.751  -10.385   23.232
   0.045  358.701  308.475   92.237   -0.895   -8.619   19.415
   0.060  357.859  301.530  107.937   -1.070   -7.626   17.298
   0.075  356.851  295.261  122.191   -1.282   -7.008   15.969
   0.090  355.645  289.414  135.512   -1.531   -6.634   15.086
   0.105  354.209  283.791  148.200   -1.814   -6.498   14.477
   0.120  352.519  278.115  160.433   -2.122   -6.853   13.991
   0.135  350.560  271.129  172.081   -2.436  -11.350   12.651
   0.150  348.336  228.896  174.008   -2.736  -89.233  -11.495
   0.165  345.866  174.785  162.643   -3.006  -37.545  -13.887
   0.180  343.182  154.756  150.488   -3.233  -13.147  -14.384
   0.195  340.323  148.097  137.893   -3.412   -3.653  -14.956
   0.210  337.331  147.121  124.708   -3.544    0.919  -15.779
   0.225  335.022  148.445  114.271   -1.808    1.539   -8.325
   0.240  334.244  149.172  110.656    0.000    0.000    0.000
\end{verbatim}

\fig{trajectory2_joint1}{RRR Joint 1 Angle and Velocity}
\fig{trajectory2_joint2}{RRR Joint 2 Angle and Velocity}
\fig{trajectory2_joint3}{RRR Joint 3 Angle and Velocity}
\fig{trajectory2_EECS}{Position of the end effector for each of the coordinates vs time}
\fig{trajectory2_EECS_3D}{RRR end effector position in 3D}
\newpage
World transforms of the robot at positions calculated for coordinate interpolated motion.
\begin{verbatim}
POSITION 1: INPUT: JOINT VARIABLES (   0.000, 326.976,  50.858,)
OUTPUT:     1.000     0.000     0.000     4.000
            0.000     0.952    -0.306     1.500
            0.000     0.306     0.952    13.000
            0.000     0.000     0.000     1.000
\end{verbatim} \pagebreak[1]\begin{verbatim}
POSITION 2: INPUT: JOINT VARIABLES ( 359.861, 324.008,  57.465,)
OUTPUT:     1.000     0.002    -0.001     4.004
           -0.002     0.931    -0.366     1.464
            0.000     0.366     0.931    12.768
            0.000     0.000     0.000     1.000
\end{verbatim} \pagebreak[1]\begin{verbatim}
POSITION 3: INPUT: JOINT VARIABLES ( 359.406, 316.558,  74.095,)
OUTPUT:     1.000     0.009    -0.005     4.014
           -0.010     0.860    -0.510     1.357
            0.000     0.510     0.860    12.071
            0.000     0.000     0.000     1.000
\end{verbatim} \pagebreak[1]\begin{verbatim}
POSITION 4: INPUT: JOINT VARIABLES ( 358.701, 308.475,  92.237,)
OUTPUT:     1.000     0.017    -0.015     4.029
           -0.023     0.758    -0.652     1.214
            0.000     0.652     0.758    11.143
            0.000     0.000     0.000     1.000
\end{verbatim} \pagebreak[1]\begin{verbatim}
POSITION 5: INPUT: JOINT VARIABLES ( 357.859, 301.530, 107.937,)
OUTPUT:     0.999     0.024    -0.028     4.043
           -0.037     0.649    -0.759     1.071
            0.000     0.760     0.650    10.214
            0.000     0.000     0.000     1.000
\end{verbatim} \pagebreak[1]\begin{verbatim}
POSITION 6: INPUT: JOINT VARIABLES ( 356.851, 295.261, 122.191,)
OUTPUT:     0.998     0.030    -0.046     4.057
           -0.055     0.537    -0.842     0.929
            0.000     0.843     0.538     9.286
            0.000     0.000     0.000     1.000
\end{verbatim} \pagebreak[1]\begin{verbatim}
POSITION 7: INPUT: JOINT VARIABLES ( 355.645, 289.414, 135.512,)
OUTPUT:     0.997     0.032    -0.069     4.071
           -0.076     0.423    -0.903     0.786
            0.000     0.906     0.424     8.357
            0.000     0.000     0.000     1.000
\end{verbatim} \pagebreak[1]\begin{verbatim}
POSITION 8: INPUT: JOINT VARIABLES ( 354.209, 283.791, 148.200,)
OUTPUT:     0.995     0.031    -0.096     4.086
           -0.101     0.308    -0.946     0.643
            0.000     0.951     0.309     7.429
            0.000     0.000     0.000     1.000
\end{verbatim} \pagebreak[1]\begin{verbatim}
POSITION 9: INPUT: JOINT VARIABLES ( 352.519, 278.115, 160.433,)
OUTPUT:     0.991     0.026    -0.128     4.100
           -0.130     0.197    -0.972     0.500
            0.000     0.980     0.199     6.500
            0.000     0.000     0.000     1.000
\end{verbatim} \pagebreak[1]\begin{verbatim}
POSITION 10: INPUT: JOINT VARIABLES ( 350.560, 271.129, 172.081,)
OUTPUT:     0.986     0.019    -0.163     4.114
           -0.164     0.117    -0.980     0.357
            0.000     0.993     0.118     5.571
            0.000     0.000     0.000     1.000
\end{verbatim} \pagebreak[1]\begin{verbatim}
POSITION 11: INPUT: JOINT VARIABLES ( 348.336, 228.896, 174.008,)
OUTPUT:     0.979     0.148    -0.138     4.129
           -0.202     0.717    -0.667     0.214
            0.000     0.681     0.733     4.643
            0.000     0.000     0.000     1.000
\end{verbatim} \pagebreak[1]\begin{verbatim}
POSITION 12: INPUT: JOINT VARIABLES ( 345.866, 174.785, 162.643,)
OUTPUT:     0.970     0.225     0.094     4.143
           -0.244     0.895     0.372     0.071
            0.000    -0.384     0.923     3.714
            0.000     0.000     0.000     1.000
\end{verbatim} \pagebreak[1]\begin{verbatim}
POSITION 13: INPUT: JOINT VARIABLES ( 343.182, 154.756, 150.488,)
OUTPUT:     0.957     0.167     0.236     4.157
           -0.289     0.552     0.782    -0.071
            0.000    -0.817     0.577     2.786
            0.000     0.000     0.000     1.000
\end{verbatim} \pagebreak[1]\begin{verbatim}
POSITION 14: INPUT: JOINT VARIABLES ( 340.323, 148.097, 137.893,)
OUTPUT:     0.942     0.093     0.324     4.171
           -0.337     0.259     0.905    -0.214
            0.000    -0.961     0.275     1.857
            0.000     0.000     0.000     1.000
\end{verbatim} \pagebreak[1]\begin{verbatim}
POSITION 15: INPUT: JOINT VARIABLES ( 337.331, 147.121, 124.708,)
OUTPUT:     0.923     0.012     0.385     4.186
           -0.385     0.029     0.922    -0.357
            0.000    -0.999     0.032     0.929
            0.000     0.000     0.000     1.000
\end{verbatim} \pagebreak[1]\begin{verbatim}
POSITION 16: INPUT: JOINT VARIABLES ( 335.022, 148.445, 114.271,)
OUTPUT:     0.906    -0.054     0.419     4.196
           -0.422    -0.115     0.899    -0.464
            0.000    -0.992    -0.127     0.232
            0.000     0.000     0.000     1.000
\end{verbatim} \pagebreak[1]\begin{verbatim}
POSITION 17: INPUT: JOINT VARIABLES ( 334.244, 149.172, 110.656,)
OUTPUT:     0.901    -0.077     0.428     4.200
           -0.435    -0.159     0.886    -0.500
            0.000    -0.984    -0.177     0.000
            0.000     0.000     0.000     1.000
\end{verbatim} \pagebreak[1]
Manipulability of the robot.
 \pagebreak[1]\begin{verbatim}
%  t(s)       ax       by       cz
   0.000   1.307    1.822    9.771
   0.015   1.307    1.994    9.539
   0.030   1.239    2.452    8.853
   0.045   1.123    2.917    7.961
   0.060   1.001    3.268    7.106
   0.075   0.873    3.536    6.305
   0.090   0.734    3.738    5.581
   0.105   0.574    3.885    4.968
   0.120   0.384    3.980    4.515
   0.135   0.165    3.987    4.302
   0.150   0.173    2.345    5.365
   0.165   0.648    1.745    5.703
   0.180   0.644    3.248    5.340
   0.195   0.757    4.027    5.290
   0.210   0.885    4.049    5.890
   0.225   0.980    3.855    6.524
   0.240   1.010    3.775    6.746
\end{verbatim}

\fig{trajectory2_ell1}{Manipulability of RRR at position 0.}
\fig{trajectory2_ell11}{Manipulability of RRR at position 10.}
\fig{trajectory2_ell17}{Manipulability of RRR at position 16.}


\newpage
Changing the time step to 0.0001 s gives a finer step of the X, Y and Z positions. It also shows where the actual peak angular joint rate is as it approaches the singularity.

RR robot for dT=0.0001s.
\fig{trajectory1_slow_joint1}{RR Joint 1 Angle and Velocity}
\fig{trajectory1_slow_joint2}{RR Joint 2 Angle and Velocity}
\fig{trajectory1_slow_ell3001}{Manipulability of RR at position 3000.}


RRR robot for dT=0.0001s.
\fig{trajectory2_slow_joint1}{RRR Joint 1 Angle and Velocity}
\fig{trajectory2_slow_joint2}{RRR Joint 2 Angle and Velocity}
\fig{trajectory2_slow_joint3}{RRR Joint 3 Angle and Velocity}
\fig{trajectory2_slow_ell1529}{Manipulability of RRR at position 1528.}


\newpage
\begin{appendices}
\section{M-Code}
\subsection{phase1.m}
This has been slightly edited since last time to work better in the phase2 workflow.
\lstinputlisting{phase1.m}

\subsection{phase2.m}
Main program for phase 2.
\lstinputlisting{phase2.m}

\subsection{JCalc.m}
Jacobian calculator.
\lstinputlisting{JCalc.m}

\subsection{Tworld.m}
Calculate $T_w$ and return it. This was pulled out of phase1 and made its own function because it was used multiple times in multiple different functions. (Such as in the Jacobian)
\lstinputlisting{Tworld.m}

\subsection{shape.m}
New shape.m function with the added RR.
\lstinputlisting{shape.m}

\subsection{manipulability.m}
Manipulability calculator.
\lstinputlisting{manipulability.m}

\end{appendices}

\end{spacing}
\end{document}