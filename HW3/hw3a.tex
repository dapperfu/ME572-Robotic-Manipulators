\documentclass{report}
% Change "article" to "report" to get rid of page number on title page
\usepackage{amsmath,amsfonts,amsthm,amssymb}
\usepackage{setspace}
\usepackage{Tabbing}
\usepackage{fancyhdr}
\usepackage{lastpage}
\usepackage{extramarks}
\usepackage{chngpage}
\usepackage{soul,color}
\usepackage{graphicx,float,wrapfig}
\usepackage{fouridx}
\usepackage{tensor}


% In case you need to adjust margins:
\topmargin=-0.45in      %
\evensidemargin=-.50in     %
\oddsidemargin=-.5in      %
\textwidth=6.5in        %
\textheight=9.0in       %
\headsep=0.25in         %

% Homework Specific Information
\newcommand{\hmwkTitle}{}
\newcommand{\hmwkDueDate}{Feb 3, 2012}
\newcommand{\hmwkClass}{ME572}
\newcommand{\hmwkNumber}{3a}
\newcommand{\hmwkAuthorName}{Jedediah Frey}

% Setup the header and footer
\pagestyle{fancy}                                                       %
\lhead{\hmwkAuthorName}                                                 %
\chead{\hmwkClass\ Homework \#\hmwkNumber}  %
\rhead{\firstxmark}                                                     %
\lfoot{\lastxmark}                                                      %
\cfoot{}                                                                %
\rfoot{Page\ \thepage\ of\ \pageref{LastPage}}                          %
\renewcommand\headrulewidth{0.4pt}                                      %
\renewcommand\footrulewidth{0.4pt}                                      %

% This is used to trace down (pin point) problems
% in latexing a document:
%\tracingall

%%%%%%%%%%%%%%%%%%%%%%%%%%%%%%%%%%%%%%%%%%%%%%%%%%%%%%%%%%%%%
% Some tools
\newcommand{\parens}[1]{\left(#1\right)}
\newcommand{\abs}[1]{\lvert#1\rvert}
\newcommand{\norm}[1]{\left|\left|#1\right|\right|}
\newcommand{\bs}[1]{\boldsymbol{#1}}

\newcommand{\enterProblemHeader}[1]{\nobreak\extramarks{#1}{#1 continued on next page\ldots}\nobreak%
                                    \nobreak\extramarks{#1 (continued)}{#1 continued on next page\ldots}\nobreak}%
\newcommand{\exitProblemHeader}[1]{\nobreak\extramarks{#1 (continued)}{#1 continued on next page\ldots}\nobreak%
                                   \nobreak\extramarks{#1}{}\nobreak}%

\newlength{\labelLength}
\newcommand{\labelAnswer}[2]
  {\settowidth{\labelLength}{#1}%
   \addtolength{\labelLength}{0.25in}%
   \changetext{}{-\labelLength}{}{}{}%
   \noindent\fbox{\begin{minipage}[c]{\columnwidth}#2\end{minipage}}%
   \marginpar{\fbox{#1}}%

   % We put the blank space above in order to make sure this
   % \marginpar gets correctly placed.
   \changetext{}{+\labelLength}{}{}{}}%

\setcounter{secnumdepth}{0}
\newcommand{\homeworkProblemName}{}%
\newcounter{homeworkProblemCounter}%
\newenvironment{homeworkProblem}[1][Problem \arabic{homeworkProblemCounter}]%
  {\stepcounter{homeworkProblemCounter}%
   \renewcommand{\homeworkProblemName}{#1}%
   \section{\homeworkProblemName}%
   \enterProblemHeader{\homeworkProblemName}}%
  {\exitProblemHeader{\homeworkProblemName}}%

\newcommand{\problemAnswer}[1]
  {\noindent\fbox{\begin{minipage}[c]{\columnwidth}#1\end{minipage}}}%

\newcommand{\problemLAnswer}[1]
  {\labelAnswer{\homeworkProblemName}{#1}}

\newcommand{\homeworkSectionName}{}%
\newlength{\homeworkSectionLabelLength}{}%
\newenvironment{homeworkSection}[1]%
  {% We put this space here to make sure we're not connected to the above.
   % Otherwise the changetext can do funny things to the other margin

   \renewcommand{\homeworkSectionName}{#1}%
   \settowidth{\homeworkSectionLabelLength}{\homeworkSectionName}%
   \addtolength{\homeworkSectionLabelLength}{0.25in}%
   \changetext{}{-\homeworkSectionLabelLength}{}{}{}%
   \subsection{\homeworkSectionName}%
   \enterProblemHeader{\homeworkProblemName\ [\homeworkSectionName]}}%
  {\enterProblemHeader{\homeworkProblemName}%

   % We put the blank space above in order to make sure this margin
   % change doesn't happen too soon (otherwise \sectionAnswer's can
   % get ugly about their \marginpar placement.
   \changetext{}{+\homeworkSectionLabelLength}{}{}{}}%

\newcommand{\sectionAnswer}[1]
  {% We put this space here to make sure we're disconnected from the previous
   % passage

   \noindent\fbox{\begin{minipage}[c]{\columnwidth}#1\end{minipage}}%
   \enterProblemHeader{\homeworkProblemName}\exitProblemHeader{\homeworkProblemName}%
   \marginpar{\fbox{\homeworkSectionName}}%

   % We put the blank space above in order to make sure this
   % \marginpar gets correctly placed.
   }%

%%%%%%%%%%%%%%%%%%%%%%%%%%%%%%%%%%%%%%%%%%%%%%%%%%%%%%%%%%%%%


%%%%%%%%%%%%%%%%%%%%%%%%%%%%%%%%%%%%%%%%%%%%%%%%%%%%%%%%%%%%%
% Make title
\title{\vspace{2in}\textmd{\textbf{\hmwkClass:\ Homework \#\hmwkNumber}}\\\normalsize\vspace{0.1in}\small{Due\ on\ \hmwkDueDate}}
\date{}
\author{\textbf{\hmwkAuthorName}}
%%%%%%%%%%%%%%%%%%%%%%%%%%%%%%%%%%%%%%%%%%%%%%%%%%%%%%%%%%%%%

\begin{document}
\begin{spacing}{1.1}
\maketitle
\newpage
% Uncomment the \tableofcontents and \newpage lines to get a Contents page
% Uncomment the \setcounter line as well if you do NOT want subsections
%       listed in Contents
%\setcounter{tocdepth}{1}
%\tableofcontents
%\newpage

% When problems are long, it may be desirable to put a \newpage or a
% \clearpage before each homeworkProblem environment

\clearpage
\begin{homeworkProblem}
Solve problem 2.6 in the reference text. Determine $^{i-1}A_i$ by inspection. These are listed in Eqn. \ref{A01} - \ref{A45}.  $^{5}A_0$ will be used to check the validity of the entire transform as it should be an identity matrix.
\begin{eqnarray}
\tensor*[^0]{A}{_1}=\begin{bmatrix}
-1&0&0&0\\ 
0&0&-1&c + e\\ 
0&-1&0&a - d\\ 
0&0&0&1\\ 
\end{bmatrix} \label{A01} \\
{}^1A_2=\begin{bmatrix}
0&-1&0&b\\ 
0&0&-1&a - d\\ 
1&0&0&0\\ 
0&0&0&1\\ 
\end{bmatrix} \\
^2A_3=\begin{bmatrix}
0&0&1&e\\ 
0&1&0&0\\ 
-1&0&0&a\\ 
0&0&0&1\\ 
\end{bmatrix} \\
^3A_4=\begin{bmatrix}
0&0&-1&d\\ 
1&0&0&0\\ 
0&-1&0&c\\ 
0&0&0&1\\ 
\end{bmatrix} \\
^4A_5=\begin{bmatrix}
0&0&-1&b\\ 
1&0&0&0\\ 
0&-1&0&d\\ 
0&0&0&1\\ 
\end{bmatrix} \label{A45} \\
^5A_0=\begin{bmatrix}
0&1&0&0\\ 
0&0&-1&a\\ 
-1&0&0&0\\ 
0&0&0&1\\ 
\end{bmatrix} \label{A50}
\end{eqnarray}
 $^0A_i$ were all calculated by matrix multiplication and listed in Eqn. \ref{A02} - \ref{A05}. Eqn. \ref{A00_eqn} was used check that all of the matricies were correct.  $^0A_0$ is the identity matrix as shown below.

\begin{eqnarray}
^0A_2=\begin{bmatrix}
0&1&0&-b\\ 
-1&0&0&c + e\\ 
0&0&1&0\\ 
0&0&0&1\\ 
\end{bmatrix} \label{A02} \\
^0A_3=\begin{bmatrix}
0&1&0&-b\\ 
0&0&-1&c\\ 
-1&0&0&a\\ 
0&0&0&1\\ 
\end{bmatrix}  \\
^0A_4=\begin{bmatrix} \\
1&0&0&-b\\ 
0&1&0&0\\ 
0&0&1&a - d\\ 
0&0&0&1\\ 
\end{bmatrix} \\
^0A_5=\begin{bmatrix} \\
0&0&-1&0\\ 
1&0&0&0\\ 
0&-1&0&a\\ 
0&0&0&1\\ 
\end{bmatrix} \label{A05} \\
^0A_0=\tensor*[^0]{A}{_1} \tensor*[^1]{A}{_2} \tensor*[^2]{A}{_3} \tensor*[^3]{A}{_4} \tensor*[^4]{A}{_5} \tensor*[^5]{A}{_0}\label{A00_eqn} \\
^0A_0=\begin{bmatrix}
1&0&0&0\\ 
0&1&0&0\\ 
0&0&1&0\\ 
0&0&0&1\\ 
\end{bmatrix} \label{A00}
\end{eqnarray}
\end{homeworkProblem}

\begin{homeworkProblem}

The joint constraint matrices are listed in Eqn. \ref{phi1}, \ref{phi2}, and \ref{phi3} and the shape matrices Eqn. \ref{T1}, \ref{T2}, and \ref{T3}.
\begin{eqnarray}
\phi_1&=&\begin{bmatrix}
\cos{\theta_1}&-\sin{\theta_1}&0&0\\ 
\sin{\theta_1}&\cos{\theta_1}&0&0\\ 
0&0&1&0\\ 
0&0&0&1\\ 
\end{bmatrix} \label{phi1}\\
T_1&=&\begin{bmatrix}
0.00&1.00&0.00&3.00\\ 
0.00&0.00&-1.00&0.00\\ 
-1.00&0.00&0.00&0.00\\ 
0.00&0.00&0.00&1.00\\ 
\end{bmatrix} \label{T1} \\
\phi_2&=&\begin{bmatrix}
1&0&0&0\\ 
0&1&0&0\\ 
0&0&1&s2\\ 
0&0&0&1\\ 
\end{bmatrix} \label{phi2} \\
T_2&=&\begin{bmatrix}
0.00&0.00&-1.00&0.00\\ 
0.00&1.00&0.00&0.00\\ 
1.00&0.00&0.00&0.00\\ 
0.00&0.00&0.00&1.00\\ 
\end{bmatrix} \label{T2} \\
\phi_3&=&\begin{bmatrix} 
\cos{\theta_3}&-\sin{\theta_3}&0&0\\ 
\sin{\theta_3}&\cos{\theta_3}&0&0\\ 
0&0&1&0\\ 
0&0&0&1\\ 
\end{bmatrix}  \label{phi3}\\
T_3&=&\begin{bmatrix}
0.00&0.00&1.00&3.00\\ 
1.00&0.00&0.00&0.00\\ 
0.00&1.00&0.00&0.00\\ 
0.00&0.00&0.00&1.00\\ 
\end{bmatrix} \label{T3}
\end{eqnarray}

To validate each of the joint constraight and shape matrices let $\theta_1=90^o$, $S_2=2"$, and $\theta_3=0^o$ (as shown at the bottom of the homework page). By inspection we can verify that the above equations are correct.

\begin{eqnarray}
{}^0A_1=\phi_1T_1&=&\begin{bmatrix}
0.00&0.00&1.00&0.00\\ 
0.00&1.00&0.00&3.00\\ 
-1.00&0.00&0.00&0.00\\ 
0.00&0.00&0.00&1.00\\ 
\end{bmatrix} \\
{}^0A_2=\phi_1T_1\phi_2T_2&=&\begin{bmatrix}
1.00&0.00&0.00&2.00\\ 
0.00&1.00&0.00&3.00\\ 
0.00&0.00&1.00&0.00\\ 
0.00&0.00&0.00&1.00\\ 
\end{bmatrix} \\
{}^0A_3=T_w&=&\begin{bmatrix}
0.00&0.00&1.00&5.00\\ 
1.00&0.00&0.00&3.00\\ 
0.00&1.00&0.00&0.00\\ 
0.00&0.00&0.00&1.00\\ 
\end{bmatrix}
\end{eqnarray}

\end{homeworkProblem}
\begin{homeworkProblem}
The complete manipulator transform $[T_M]$ is shown in Eqn. \ref{Tw} and reduced with trig identities to Eqn. \ref{Tw_simp}.
\begin{eqnarray}
T_w=\begin{bmatrix}
\cos{\theta_1}\cos{\theta_3}-\sin{\theta_1}\sin{\theta_3}&0&\cos{\theta_1}\sin{\theta_3}+\cos{\theta_3}\sin{\theta_1}&3\cos{\theta_1}+3\cos{\theta_1}\sin{\theta_3}+3\cos{\theta_3}\sin{\theta_1}+s2\sin{\theta_1}\\
\cos{\theta_1}\sin{\theta_3}+\cos{\theta_3}\sin{\theta_1}&0&\sin{\theta_1}\sin{\theta_3}-\cos{\theta_1}\cos{\theta_3}&3\sin{\theta_1}-3\cos{\theta_1}\cos{\theta_3}+3\sin{\theta_1}\sin{\theta_3}-s2\cos{\theta_1}\\
0&1&0&0\\
0&0&0&1\\
\end{bmatrix} \label{Tw} \\
T_w=\begin{bmatrix}
cos(theta1 + theta3)&0&sin(theta1 + theta3)&3*sin(theta1 + theta3) + 3*cos(theta1) + s2*sin(theta1)\\ 
sin(theta1 + theta3)&0&-cos(theta1 + theta3)&3*sin(theta1) - 3*cos(theta1 + theta3) - s2*cos(theta1)\\ 
0&1&0&0\\ 
0&0&0&1\\ 
\end{bmatrix} \label{Tw_simp}
\end{eqnarray}
\end{homeworkProblem}
\end{spacing}
\end{document}